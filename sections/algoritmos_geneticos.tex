\section{Algoritmos genéticos}

Algoritmos genéticos (AGs) são uma meta-heurística baseada em princípios de seleção natural para obtenção de boas soluções para problemas. As características dos AGs os tornam excelentes ferramentas para vários propósitos diferentes. Nesta seção, são descritos os passos que todos os AGs devem seguir, e são apresentadas aplicações destes algoritmos.

\subsection{Estrutura de algoritmos genéticos}

Os AGs simulam mecanismos evolutivos para obter soluções para problemas. Para tal, os AGs dependem de uma operação de conversão de soluções para cromossomos e vice-versa. Um cromossomo é uma estrutura de dados que contém as informações necessárias para reconstruir e avaliar uma solução.